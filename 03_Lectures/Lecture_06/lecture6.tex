\documentclass[
    aspectratio=169, 
    usepdftitle=false, 
    xcolor={dvipsnames},
    hyperref={
        colorlinks,
        linkcolor=black,
        urlcolor=blue}
    ]{beamer}
\usetheme{Madrid}
\usepackage{graphicx}
\usepackage{listings}
\usepackage{soul}
\usepackage{amsmath}
\lstset{
    numbers=left,
    xleftmargin=2em,
    frame=single,
    framexleftmargin=0em,
    basicstyle=\footnotesize\ttfamily,
    xleftmargin=.075\textwidth,
    xrightmargin=.075\textwidth,
    numberstyle=\footnotesize\ttfamily,
    upquote=true,
    framesep=10pt,
    numbersep=20pt,
    keywordstyle=\bfseries,
    stringstyle=\textit,
    showstringspaces=false,
    columns=fixed,
    breaklines=true,
}

\lstdefinestyle{output}{
    numbers=none,
    xleftmargin=2em,
    frame=single,
    framexleftmargin=0em,
    basicstyle=\footnotesize\ttfamily,
    xleftmargin=.075\textwidth,
    xrightmargin=.075\textwidth,
    numberstyle=\footnotesize\ttfamily,
    upquote=true,
    framesep=10pt,
    numbersep=20pt,
    keywordstyle=\bfseries,
    stringstyle=\textit,
    showstringspaces=false,
    columns=fixed,
    breaklines=true,
    backgroundcolor=\color{lightgray},
}
% \usepackage{xcolor}

\hypersetup{colorlinks,urlcolor=blue}
\addtobeamertemplate{headline}{\hypersetup{linkcolor=.}}{}
\addtobeamertemplate{footline}{\hypersetup{linkcolor=.}}{}

\definecolor{Light}{gray}{.90}
\sethlcolor{Light}

\let\OldTexttt\texttt
\renewcommand{\texttt}[1]{\OldTexttt{\hl{#1}}}% will affect all \texttt

\title[Introduction to Python]{Introduction to Python}
\subtitle{Lecture 6: Git and Collaboration}
\author{Daniel Kadyrov}
\date{July 4th, 2023}

\begin{document}

\begin{frame}
    \titlepage
\end{frame}

\begin{frame}{Git}
    \begin{itemize}
        \item Git is a version control system that lets you manage and keep track of your source code history.
        \item Git is a distributed version control system, meaning that the entire codebase and history is available on every developer's computer, which allows for easy branching and merging.
        \item Git is the most commonly used version control system, and is used as the basis for many other services and tools.
    \end{itemize}
\end{frame}

\begin{frame}{Cloning}
    \begin{itemize}
        \item Cloning is the process of downloading or copying a repository to the destination location from the server/source.
        \item Cloning a repository pulls down a full copy of all the repository data that GitHub has at that point in time, including all versions of every file and folder for the project.
        \item You can clone your existing repository or clone another person's existing repository to contribute to a project.
    \end{itemize}
\end{frame}

\begin{frame}{Pulling}
    \begin{itemize}
        \item Pulling is the process of downloading and merging changes from a remote repository into your local repository.
        \item You can think of this as the opposite of pushing.
        \item Pulling is a combination of \texttt{git fetch} and \texttt{git merge}.
    \end{itemize}
\end{frame}

\begin{frame}{Pushing}
    \begin{itemize}
        \item Pushing is the process of uploading your local changes to a remote repository.
        \item It's the counterpart to \texttt{git pull}, but whereas fetching imports commits to local branches, pushing exports commits to remote branches.
        \item This has the potential to overwrite changes, so you need to be careful.
    \end{itemize}
\end{frame}

\begin{frame}{Committing}
    \begin{itemize}
        \item Committing is the process of creating a snapshot of your changes.
        \item Git compares the committed snapshots and determines what has changed since the last commit.
        \item You can think of this as a save point in a game.
    \end{itemize}
\end{frame}

\begin{frame}{Branching}
    \begin{itemize}
        \item Branching is the process of creating copies of a repository in order to develop features or make bug fixes isolated from the main codebase.
        \item Each repository has one default branch, and can have multiple other branches.
        \item You can merge a branch into another branch using a pull request.
    \end{itemize}
\end{frame}

\begin{frame}{Git Workflow}
    \begin{itemize}
        \item Create a new branch.
        \item Add commits.
        \item Open a pull request.
        \item Discuss and review your code.
        \item Deploy.
        \item Merge.
    \end{itemize}

    \begin{lstlistings}[language=bash]
        git checkout -b new-branch
        git add .
        git commit -m "message"
        git push origin new-branch
    \end{lstlistings}
\end{frame}

\end{document}