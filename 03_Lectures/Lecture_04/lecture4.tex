\documentclass[
    aspectratio=169, 
    usepdftitle=false, 
    xcolor={dvipsnames},
    hyperref={
        colorlinks,
        linkcolor=black,
        urlcolor=blue}
    ]{beamer}
\usetheme{Madrid}
\usepackage{graphicx}
\usepackage{listings}
\usepackage{soul}
\usepackage{amsmath}
\lstset{
    numbers=left,
    xleftmargin=2em,
    frame=single,
    framexleftmargin=0em,
    basicstyle=\footnotesize\ttfamily,
    xleftmargin=.075\textwidth,
    xrightmargin=.075\textwidth,
    numberstyle=\footnotesize\ttfamily,
    upquote=true,
    framesep=10pt,
    numbersep=20pt,
    keywordstyle=\bfseries,
    stringstyle=\textit,
    showstringspaces=false,
    columns=fixed,
    breaklines=true,
}

\lstdefinestyle{output}{
    numbers=none,
    xleftmargin=2em,
    frame=single,
    framexleftmargin=0em,
    basicstyle=\footnotesize\ttfamily,
    xleftmargin=.075\textwidth,
    xrightmargin=.075\textwidth,
    numberstyle=\footnotesize\ttfamily,
    upquote=true,
    framesep=10pt,
    numbersep=20pt,
    keywordstyle=\bfseries,
    stringstyle=\textit,
    showstringspaces=false,
    columns=fixed,
    breaklines=true,
    backgroundcolor=\color{lightgray},
}
% \usepackage{xcolor}

\hypersetup{colorlinks,urlcolor=blue}
\addtobeamertemplate{headline}{\hypersetup{linkcolor=.}}{}
\addtobeamertemplate{footline}{\hypersetup{linkcolor=.}}{}

\definecolor{Light}{gray}{.90}
\sethlcolor{Light}

\let\OldTexttt\texttt
\renewcommand{\texttt}[1]{\OldTexttt{\hl{#1}}}% will affect all \texttt

\title[Introduction to Python]{Introduction to Python}
\subtitle{Lecture 4: Packages}
\author{Daniel Kadyrov}
\date{July 7th, 2023}

\begin{document}

\begin{frame}
    \titlepage
\end{frame}

\begin{frame}{Python Packages}
    \begin{itemize}
    \item One of the most powerful features of Python is the availability of a large number of libraries and packages that can be imported into your code.
    \item These packages are open-source and managed by the Python community.
    \item The Python Package Index (PyPI) is a repository of software for the Python programming language. Packages are usually installed from the Python Package Index using the \texttt{pip} command.
    \end{itemize}
\end{frame}

\begin{frame}[fragile]{Installing a package}
    The following command installs the \texttt{numpy} package. This command is performed in the terminal, not in the Python interpreter. Numpy is a package for scientific computing in Python that extends the functionality of Python lists to arrays.

    \begin{lstlisting}[language=bash]
pip install numpy
    \end{lstlisting}

    To use the package in your code, you need to import it. The following code snippet imports the \texttt{numpy} package and assigns it the alias \texttt{np}.

    \begin{lstlisting}[language=Python]
import numpy as np
    \end{lstlisting}
\end{frame}

\begin{frame}[fragile]{Numpy}
    \begin{itemize}
        \item Numpy is a package for scientific computing in Python that extends the functionality of Python lists to arrays.
        \item Numpy arrays are more efficient than Python lists.
        \item Numpy arrays are homogeneous, i.e. they can only contain elements of the same type.
        \item Numpy arrays can be multidimensional.
        \item Numpy arrays can be created from Python lists using the \texttt{array()} function.
    \end{itemize}
\end{frame}
\begin{frame}[fragile]{Numpy}

    The following example shows some of the basic operations that can be performed on numpy arrays. The \texttt{linspace} function creates an array of 5 elements between 10 and 14. The \texttt{arange} function creates an array of 5 elements between 1 and 5. The \texttt{add} function adds the two arrays element-wise.

    \begin{columns}
        \begin{column}{0.5\textwidth}
\begin{lstlisting}[language=Python]
import numpy as np
a = np.linspace(10, 14, 5)
print(a)    
b = np.arange(1, 6)
print(b)
print(np.add(a,b))
\end{lstlisting}
\end{column}
\begin{column}{0.4\textwidth}
\begin{lstlisting}[style=output]
[10. 11. 12. 13. 14.]
[1 2 3 4 5]
[11. 13. 15. 17. 19.]
\end{lstlisting}
\end{column}
\end{columns}
\end{frame}

\begin{frame}{Pandas}
    \begin{itemize}
        \item Pandas is a Python package for data manipulation and analysis.
        \item Pandas provides data structures and functions that make working with structured data easier.
        \item Pandas is built on top of Numpy.
        \item Pandas provides two data structures: Series and DataFrame.
        \item A Series is a one-dimensional array of indexed data.
        \item A DataFrame is a two-dimensional array of indexed data.
    \end{itemize}
\end{frame}

\begin{frame}{Pandas}
    \begin{itemize}
        \item Pandas provides functions to read data from different file formats, such as CSV, Excel, JSON, HTML, etc.
        \item Pandas provides functions to write data to different file formats, such as CSV, Excel, JSON, HTML, etc.
        \item Pandas provides functions to manipulate data, such as merging, reshaping, selecting, etc.
        \item Pandas provides functions to perform statistical analysis on data.
        \item Pandas provides functions to visualize data.
    \end{itemize}
\end{frame}

\begin{frame}[fragile]{Pandas}
    The following example shows how to create a DataFrame from a CSV file. The \texttt{read\_csv} function reads the CSV file and creates a DataFrame. The \texttt{head} function displays the first 5 rows of the DataFrame.

    \begin{columns}
        \begin{column}{0.45\textwidth}
\begin{lstlisting}[language=Python]
import pandas as pd
df = pd.read_csv('data.csv')
df.head()
\end{lstlisting}
\end{column}
\begin{column}{0.45\textwidth}
\begin{lstlisting}[style=output]
   id  age  weight  height
0   1   22      65     170
1   2   25      70     175
2   3   28      75     180
3   4   31      80     185
4   5   34      85     190
\end{lstlisting}
\end{column}
\end{columns}
\end{frame}

\begin{frame}[fragile]{Pandas}
    There are a multitude of functions available from Panda to manipulate data. The following example shows how to select a subset of the data. The \texttt{loc} function selects rows and columns by label. The \texttt{iloc} function selects rows and columns by position.

    \begin{columns}
        \begin{column}{0.45\textwidth}
\begin{lstlisting}[language=Python]
import pandas as pd
df = pd.read_csv('data.csv')
df.loc[0:2, ['age', 'weight']]
df.iloc[0:2, 1:3]
\end{lstlisting}
\end{column}
\begin{column}{0.45\textwidth}
\begin{lstlisting}[style=output]
   age  weight
0   22      65
1   25      70
2   28      75
   age  weight
0   22      65
1   25      70
\end{lstlisting}
\end{column}
\end{columns}
\end{frame}

\begin{frame}[fragile]{Pandas}
    The following example shows how to perform statistical analysis on data. The \texttt{describe} function computes summary statistics for numerical columns. The \texttt{value\_counts} function counts the number of occurrences of each value in a column.

\begin{lstlisting}[language=Python]
import pandas as pd
df = pd.read_csv('data.csv')
df.describe()
df['age'].value_counts()
\end{lstlisting}
\end{frame}

\begin{frame}[fragile]{Pandas}
    The following example shows how to visualize data. The \texttt{plot} function plots the data in a DataFrame. The \texttt{plot} function can be used to plot different types of plots, such as line plots, bar plots, pie plots, scatter plots, etc.

\begin{lstlisting}[language=Python]
import pandas as pd
df = pd.read_csv('data.csv')
df.plot()
df.plot(kind='bar')
df.plot(kind='pie')
df.plot(kind='scatter', x='age', y='weight')
\end{lstlisting}
\end{frame}

\begin{frame}{Graphing}
    There are several plotting libraries for Python. The most popular ones are Matplotlib, Seaborn, and Plotly.
\end{frame}

\begin{frame}[fragile]{Writing Packages}
    \begin{itemize}
        \item A package is a collection of Python modules.
        \item A package is a directory containing \texttt{\_\_init\_\_.py} file.
        \item This file can be empty, and it indicates that the directory it contains is a Python package, so it can be imported the same way a module can be imported.
        \item A package can contain subpackages, which are subdirectories containing again a \texttt{\_\_init\_\_.py} file, and submodules, which are Python scripts like any other Python modules.
        \item A package can also contain \texttt{c-extensions}, which are compiled C code.
    \end{itemize}
\end{frame}
\end{document}