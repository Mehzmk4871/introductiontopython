\documentclass[
    aspectratio=169, 
    usepdftitle=false, 
    xcolor={dvipsnames},
    hyperref={
        colorlinks,
        linkcolor=black,
        urlcolor=blue}
    ]{beamer}
\usetheme{Madrid}
\usepackage{graphicx}
\usepackage{listings}
\usepackage{soul}
\usepackage{amsmath}
\lstset{
    numbers=left,
    xleftmargin=2em,
    frame=single,
    framexleftmargin=0em,
    basicstyle=\footnotesize\ttfamily,
    xleftmargin=.075\textwidth,
    xrightmargin=.075\textwidth,
    numberstyle=\footnotesize\ttfamily,
    upquote=true,
    framesep=10pt,
    numbersep=20pt,
    keywordstyle=\bfseries,
    stringstyle=\textit,
    showstringspaces=false,
    columns=fixed,
    breaklines=true,
}

\lstdefinestyle{output}{
    numbers=none,
    xleftmargin=2em,
    frame=single,
    framexleftmargin=0em,
    basicstyle=\footnotesize\ttfamily,
    xleftmargin=.075\textwidth,
    xrightmargin=.075\textwidth,
    numberstyle=\footnotesize\ttfamily,
    upquote=true,
    framesep=10pt,
    numbersep=20pt,
    keywordstyle=\bfseries,
    stringstyle=\textit,
    showstringspaces=false,
    columns=fixed,
    breaklines=true,
    backgroundcolor=\color{lightgray},
}
% \usepackage{xcolor}

\hypersetup{colorlinks,urlcolor=blue}
\addtobeamertemplate{headline}{\hypersetup{linkcolor=.}}{}
\addtobeamertemplate{footline}{\hypersetup{linkcolor=.}}{}

\definecolor{Light}{gray}{.90}
\sethlcolor{Light}

\let\OldTexttt\texttt
\renewcommand{\texttt}[1]{\OldTexttt{\hl{#1}}}% will affect all \texttt

\title[Introduction to Python]{Introduction to Python Objects and Expressions}
\subtitle{Lecture 1: Introduction to Python}
\author{Daniel Kadyrov}
\date{July 4th, 2023}

\begin{document}

\begin{frame}
    \titlepage
\end{frame}

\begin{frame}{What is Python?}
    \begin{itemize}
        \item Python is a high-level, interpreted, general-purpose programming language.
        \item Python supports multiple programming paradigms, such as object-oriented, imperative, functional, and procedural.
        \item Python has a large and comprehensive standard library that provides built-in modules for various tasks, such as data structures, file handling, networking, etc.
    \end{itemize}
\end{frame}

\begin{frame}{Where is Python used?}
    Python is used in a wide variety of fields, including:
    \begin{itemize}
        \item Web development (server-side) with frameworks such as Django and Flask
        \item Software development (build control, testing, etc.)
        \item Mathematics and science
        \item System scripting (automation, etc.)
        \item Internet of Things (Raspberry Pi, MicroPython, etc.)
        \item \textbf{Data science (machine learning, data analysis, data visualization)}
    \end{itemize}
\end{frame}

\begin{frame}{Running Python}
    \begin{itemize}
        \item Python is an interpreted language, which means that Python code is executed line by line.
        \item Python programs, also called scripts, are plain text files with the .py extension. You can run the programs through the following methods:
              \begin{itemize}
                  \item Running Python scripts from the command line or terminal by typing \texttt{python script.py} where script.py is the name of the script.
                  \item Within VSCode with the jupyter notebook extension or the debug tool
              \end{itemize}
    \end{itemize}
\end{frame}

\begin{frame}[fragile]{Hello World!}
    The first code you will run in almost every programming course is Hello World. This is a simple program that prints Hello World! to the screen. In Python, this can be done with a single line of code:

    \begin{lstlisting}[language=Python]
print("Hello World!")
    \end{lstlisting}

    \begin{itemize}
        \item The print() function prints the specified message to the screen.
        \item The message can be a string, or any other object, the print() function will try to convert it to a string.
    \end{itemize}
\end{frame}

\begin{frame}{Python Objects}
    \begin{itemize}
        \item In Python, everything is an object.
        \item An object is a collection of data and methods that operate on that data.
        \item An object has a type that determines what kind of data it can store and what methods it can use (string, integer, float, etc.)
        \item For example, a string object can store a sequence of characters and has methods for manipulating strings, such as upper(), lower(), replace(), etc.
    \end{itemize}
\end{frame}

\begin{frame}{Python Expressions}
    \begin{itemize}
        \item An expression is a piece of code that evaluates to a value.
        \item An expression can consist of literals, variables, operators, function calls, etc.
        \item For example, 2 + 3 is an expression that evaluates to 5.
        \item Expressions can be used to assign values to variables, pass arguments to functions, return values from functions, etc.
    \end{itemize}
\end{frame}

\begin{frame}[fragile]{Python Comments}
    \begin{itemize}
        \item Comments are used to explain Python code.
        \item Comments are ignored by the Python interpreter.
        \item Comments can be used to prevent execution when testing code.
        \item Comments start with a \texttt{\#} and end at the end of the line.
        \item Comments can be placed on a line by themselves, or at the end of a line of code.
    \end{itemize}
    \begin{lstlisting}[language=Python]
# This is a comment
print("Hello World!") # This is also a comment
    \end{lstlisting}
\end{frame}


\begin{frame}[fragile]{Python Variables}
    \begin{itemize}
        \item Variables are used to store data in memory.
        \item Variables are created when they are assigned a value.
        \item Variables can be assigned a new value at any time.
        \item Variables are assigned using the assignment operator \texttt{=}.
        \item Variable names can contain letters, numbers, and underscores, but cannot start with a number.
        \item \href{https://peps.python.org/pep-0008/}{PEP8} style guide recommends using lowercase letters and underscores for variable names.
    \end{itemize}
    \begin{columns}
        \begin{column}{0.5\textwidth}
            \begin{lstlisting}
x = 5
y = 10
z = x + y
print(z)
z = y - x
print(z)
    \end{lstlisting}
        \end{column}
        \begin{column}{0.4\textwidth}
\begin{lstlisting}[style=output]
15
5
\end{lstlisting}
        \end{column}
    \end{columns}
\end{frame}

\begin{frame}[fragile]{Python Data Types}
    Python has several built-in data types, including:
    \begin{itemize}
        \item \textbf{Numeric types:} int, float, complex
        \item \textbf{Boolean type:} bool
        \item \textbf{Sequence types:} list, tuple, range
        \item \textbf{Mapping type:} dict
        \item \textbf{Set types:} set, frozenset
        \item \textbf{String type:} str
    \end{itemize}
    The variable type can be checked with the \texttt{type()} function.
\end{frame}



\begin{frame}[fragile]{Data Types}
    \framesubtitle{Numeric}
    \begin{itemize}
        \item Python has three numeric types: int, float, and complex.
        \item Integers are whole numbers, positive or negative, without decimals, of unlimited length.
        \item Floats are numbers with a decimal point and can be used to represent real numbers.
        \item Complex numbers are written with a "j" as the imaginary part.
    \end{itemize}
    \begin{lstlisting}[language=Python]
x = 1    # int
y = 2.8  # float
z = 1j   # complex
\end{lstlisting}
\end{frame}

\begin{frame}[fragile]{Mathematical Expressions}
    Python supports the following mathematical operators:
    \begin{columns}
        \begin{column}{0.5\textwidth}
            \begin{lstlisting}
x = 5
y = 2
print(x + y) # Addition
print(x - y) # Subtraction
print(x * y) # Multiplication
print(x / y) # Division
print(x % y) # Modulus
print(x ** y) # Exponentiation
print(x // y) # Floor division
    \end{lstlisting}
        \end{column}
        \begin{column}{0.4\textwidth}
\begin{lstlisting}[style=output]
7
3
10
2.5
1
25
2
\end{lstlisting}
\end{column}
\end{columns}
\end{frame}

\begin{frame}[fragile]{Data Types}
    \framesubtitle{Boolean}
    \begin{itemize}
        \item Boolean values are the two constant objects False and True.
        \item Boolean values are used to evaluate conditions.
        \item The comparison operators \texttt{==}, \texttt{!=}, \texttt{>}, \texttt{<}, \texttt{>=}, \texttt{<=} return boolean values.
        \item The boolean operators \texttt{and}, \texttt{or}, and \texttt{not} are used to combine boolean values.
    \end{itemize}
    \begin{columns}
        \begin{column}{0.5\textwidth}
            \begin{lstlisting}[language=Python]
x = True
y = False
print(x and y)
print(x or y)
print(not x)
\end{lstlisting}
        \end{column}
        \begin{column}{0.4\textwidth}
\begin{lstlisting}[style=output]
False
True
False
\end{lstlisting}
        \end{column}
    \end{columns}
\end{frame}

\begin{frame}[fragile]{Data Types}
    \framesubtitle{Sequence}
    \begin{itemize}
        \item Python has three sequence types: list, tuple, and range.
        \item Python is a zero-indexed language, meaning the first item in a sequence is at index 0.
        \item Lists are ordered and changeable sequences of items. They are the most commonly used sequence type.
        \item Lists have several methods for manipulating them including:
              \begin{itemize}
                  \item \texttt{append()} to add an item to the end of the list.
                  \item \texttt{insert()} to insert an item at a specified index.
                  \item \texttt{remove()} to remove an item from the list.
                  \item \texttt{pop()} to remove an item at a specified index.
              \end{itemize}
        \item Lists can also be indexed and sliced like strings through the use of square brackets \texttt{[]}
        \item There are many more methods available for lists available in the Python documentation at \url{https://docs.python.org/3/tutorial/datastructures.html#more-on-lists}
    \end{itemize}
\end{frame}

\begin{frame}[fragile]{Data Types}
    \framesubtitle{Sequence Examples}
    The following code demonstrates some of the methods available for lists:
    \begin{columns}
        \begin{column}{0.5\textwidth}
            \begin{lstlisting}[language=Python]
    x = [1, 2, 3, 4, 5]
    print(x)
    x.append(6)
    print(x)
    print(x[0])
    print(x[1])
    print(x[-1])
    print(x[-2])
    x[0] = 0
    print(x)
    print(len(x))
    \end{lstlisting}
        \end{column}
        \begin{column}{0.4\textwidth}
            \begin{lstlisting}[style=output]
[1, 2, 3, 4, 5]
[1, 2, 3, 4, 5, 6]
1
2
6
5
[0, 2, 3, 4, 5, 6]
6
            \end{lstlisting}
        \end{column}
    \end{columns}
\end{frame}

\begin{frame}[fragile]{Data Types}
    \framesubtitle{Dictionaries}
    \begin{itemize}
        \item Dictionaries are unordered, changeable, and indexed collections of key-value pairs.
        \item Dictionaries are indexed by keys, which can be any immutable type.
        \item Dictionaries are created using curly brackets \texttt{\{\}} and key-value pairs separated by commas.
        \item Dictionaries have several methods for manipulating them including:
              \begin{itemize}
                  \item \texttt{get()} to get the value of a specified key.
                  \item \texttt{pop()} to remove an item with a specified key.
                  \item \texttt{keys()} to get a list of all the keys in the dictionary.
                  \item \texttt{values()} to get a list of all the values in the dictionary.
              \end{itemize}
        \item There are many more methods available for dictionaries available in the Python documentation at \url{https://docs.python.org/3/library/stdtypes.html#mapping-types-dict}
    \end{itemize}
\end{frame}
\begin{frame}[fragile]{Data Types}
    \framesubtitle{Dictionaries Examples}
    The following code demonstrates some of the methods available for dictionaries:

    \begin{columns}
        \begin{column}{0.3\textwidth}
            \begin{lstlisting}[language=Python]
x = {
    "name": "John",
    "age": 36,
    "country": "Norway"
}
print(x)
print(x["name"])
print(x.get("age"))
x["age"] = 37
print(x)
print(x.keys())
print(x.values())
    \end{lstlisting}
        \end{column}
        \begin{column}{0.6\textwidth}
            \begin{lstlisting}[style=output]
{'name': 'John', 'age': 36, 'country': 'Norway'}
John
36
{'name': 'John', 'age': 37, 'country': 'Norway'}
dict_keys(['name', 'age', 'country'])
dict_values(['John', 37, 'Norway'])
\end{lstlisting}                  
        \end{column}
    \end{columns}
\end{frame}
\begin{frame}[fragile]{Data Types}
    \framesubtitle{String}
    \begin{itemize}
        \item Strings in Python are sequences of characters enclosed in single or double quotes.
        \item A multitude of methods are available for manipulating strings including:
              \begin{itemize}
                  \item \texttt{upper()} and \texttt{lower()} to convert the string to uppercase or lowercase.
                  \item \texttt{replace()} to replace a substring with another substring.
                  \item \texttt{split()} to split the string into a list of substrings.
                  \item \texttt{join()} to join a list of strings into one string.
                  \item Strings can also be indexed and sliced like lists through the use of square brackets \texttt{[]}
              \end{itemize}
        \item There are many more methods available for strings available in the Python documentation at \url{https://docs.python.org/3/library/stdtypes.html#string-methods}
    \end{itemize}
\end{frame}

\begin{frame}[fragile]{Data Types}
    \framesubtitle{String: Examples}

    The following code demonstrates some of the methods available for strings:

    \begin{columns}
        \begin{column}{0.5\textwidth}

            \begin{lstlisting}[language=Python]
s = "Hello World!"
print(s)
print(s.upper())
print(s.lower())
print(s.replace("World", "Python"))
print(s.split(" "))
print(" ".join(["Hello", "World!"]))
print(s[0])
print(s[0:5])
            \end{lstlisting}

        \end{column}
        \begin{column}{0.4\textwidth}
\begin{lstlisting}[style=output]
Hello World!
HELLO WORLD!
hello world!
Hello Python!
['Hello', 'World!']
Hello World!
H
Hello
            \end{lstlisting}
        \end{column}
    \end{columns}
\end{frame}
\end{document}